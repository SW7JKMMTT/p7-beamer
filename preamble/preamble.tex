%%% Initial things %%%
% Increase number of dimen registers
\usepackage{etex}
% Fix various issues with LaTeX2e
\usepackage{fixltx2e}
% Font package
\usepackage{fourier}


%%% Translations and character encodings %%%
% Enable use of several characters, including æ, ø and å
\usepackage[utf8]{inputenc}
% Danish language
\usepackage[danish]{babel}
% Use PostScript fonts instead of bitmap ones. Also does other stuff.
\usepackage[T1]{fontenc}
% Various LaTeX symbols
\usepackage{latexsym}
% Wider selection of colours
\usepackage{xcolor}
\definecolor{aaublue}{gray}{0}
\definecolor{bluekeywords}{gray}{0}
\definecolor{greencomments}{gray}{0.5}
\definecolor{redstrings}{gray}{0.3}
\definecolor{codebg}{HTML}{EFEFEF}
\definecolor{codefg}{HTML}{000000}
\definecolor{part}{HTML}{34495E}
\definecolor{numbers}{HTML}{34495E}
\definecolor{smartdiagram1}{HTML}{1ABC9C}
\definecolor{smartdiagram2}{HTML}{2ECC71}
\definecolor{smartdiagram3}{HTML}{3498db}
\definecolor{smartdiagram4}{HTML}{9b59b6}
\definecolor{smartdiagram5}{HTML}{E74C3C}
\definecolor{smartdiagram6}{HTML}{F1C40F}
\definecolor{smartdiagram7}{HTML}{E67E22}
\definecolor{diagramDark}{HTML}{19B5FE}
\definecolor{diagramLight}{HTML}{6BB9F0}

\definecolor{tableGoodLight}{HTML}{87D37C}
\definecolor{tableGoodDark}{HTML}{26A65B}
\definecolor{tableBadLight}{HTML}{EC644B}
\definecolor{tableBadDark}{HTML}{EF4836}
% Improved element justification
\usepackage{ragged2e}
% Font improvements
\usepackage{fix-cm}
% Enables various forms of lines, like double-underlining (\uuline{})
\usepackage{ulem}
% Sets the tolerance for distance between words, determining when to hyphenate.
\pretolerance=2500


%%% Figures and tables (Floats) %%%
% Enable multi-rows and -columns
\usepackage{multirow}
\usepackage{multicol}
% Double, horizontal lines
\usepackage{hhline}
% Enables coloured tables
\usepackage{colortbl}
% Prettier tables
\usepackage{booktabs}


%%% Mathematic formulas %%%
% AMS math
\usepackage{amsmath}
\usepackage{amssymb}
% Extra fonts (for math, I think)
\usepackage{stmaryrd}
\usepackage{wasysym}
% Access text symbols
\usepackage{textcomp}
% Extend AMS
\usepackage{mathtools}
\usepackage{cancel}


%%% Graphics %%%
% Various image-commands
\usepackage{eso-pic}
% Use JPEG and PNG images
\usepackage{graphicx}

\definecolor{listback}{rgb}{0.95,0.95,0.92}
\definecolor{lstkeywords}{rgb}{0.10,0.10,0.50}
\definecolor{lstrule}{rgb}{0.60, 0.60, 0.60}
\definecolor{lstoperator}{rgb}{0.10,0.50,0.10}
\definecolor{diagramLight}{HTML}{6BB9F0}

\usepackage{adjustbox}
\usepackage{listings}
\usepackage{lstautogobble}
\lstset{%
  autogobble,
  backgroundcolor=\color{listback},   % choose the background color; you must add \usepackage{color} or \usepackage{xcolor}
  basicstyle=\tiny\ttfamily,        % the size of the fonts that are used for the code
  captionpos=b,                    % sets the caption-position to bottom
  deletekeywords={},            % if you want to delete keywords from the given language
  escapeinside={<@}{@>},          % if you want to add LaTeX within your code
  extendedchars=false,              % lets you use non-ASCII characters; for 8-bits encodings only, does not work with UTF-8
  frame=none,                    % adds a frame around the code
  keepspaces=true,                 % keeps spaces in text, useful for keeping indentation of code (possibly needs columns=flexible)
  keywordstyle=\bfseries\color{lstkeywords},       % keyword style
  morekeywords={define, local, case, do, anywherein, forever, mod, continue, break},            % if you want to add more keywords to the set
  rulecolor=\color{lstrule},         % if not set, the frame-color may be changed on line-breaks within not-black text (e.g. comments (green here))
  showspaces=false,                % show spaces everywhere adding particular underscores; it overrides 'showstringspaces'
  showstringspaces=false,          % underline spaces within strings only
  showtabs=false,                  % show tabs within strings adding particular underscores
  numbers=left,
  numbersep=5pt,
  stepnumber=1,                    % the step between two line-numbers. If it's 1, each line will be numbered
  stringstyle=\bfseries\color{redstrings},     % string literal style
  tabsize=4,                       % sets default tabsize to 2 spaces
  columns=fullflexible,
}
%%% References, bibtex and URLs %%%
% Post URLs. Allows breaking at hyphens to help avoid long links.
\usepackage{url}
% Better cross references
\usepackage[danish]{varioref}
% Define a new 'leo' style for URL package, that will use a smaller font
\makeatletter
\def\url@leostyle{%
  \@ifundefined{selectfont}{\def\UrlFont{\sf}}{\def\UrlFont{\small\ttfamily}}
}
\makeatother
% And of course, use this new style
\urlstyle{leo}

%This does not work for me for some reason --Caspar
%%% Algorithmicx (Pseudocode) %%%
%\usepackage{algorithm}
%\usepackage{algpseudocode}
%\algrenewcommand\algorithmicfunction{\textbf{method}}
%\alglanguage{pseudocode}
%\newcommand\Fontvi{\fontsize{10}{7.2}\selectfont}


\hypersetup{pdfstartview={Fit}}
%%%%%%%%%%%%%%%%%%%%%%%%%%%%%%%%%%%%%%%%%%%%%%%%
%Flowchart
\usepackage{tikz}
\usepackage{packages/tikz-uml}
\usetikzlibrary{matrix, shapes.multipart, fit, arrows.meta, shapes.geometric, arrows, positioning, backgrounds, calc, patterns}
\tikzumlset{
	fill class = lightgray!20,
	fill state = white,
}
%%%%%%%%%%%%%%%%%%%%%%%%%%%%%%%%%%%%%%%%%%%%%%%%

\usepackage{listings}
\lstset{escapeinside={<@}{@>}}

\usepackage{anyfontsize}
\setbeamerfont{caption}{size=\tiny}

\usepackage{units}

\usepackage{minted}
\newminted[java2]{java}{frame=leftline, framesep=1mm,  fontsize=\footnotesize, baselinestretch=1.1, autogobble}
\newminted[xmlblock]{xml}{frame=leftline, framesep=2mm, linenos, fontsize=\footnotesize, baselinestretch=1.1, autogobble}
\newminted[bashblock]{shell}{frame=leftline, framesep=2mm, linenos, fontsize=\footnotesize, baselinestretch=1.1, autogobble, breaklines, breakanywhere}
\newmintedfile{java}{frame=leftline, framesep=2mm, linenos, fontsize=\footnotesize, baselinestretch=1.1}
\usemintedstyle{tango}

\usepackage{tikz}
\usetikzlibrary{backgrounds}
\makeatletter

\tikzset{%
  fancy quotes/.style={
    text width=\fq@width pt,
    align=justify,
    inner sep=1em,
    anchor=north west,
    minimum width=\linewidth,
  },
  fancy quotes width/.initial={.8\linewidth},
  fancy quotes marks/.style={
    scale=8,
    text=white,
    inner sep=0pt,
  },
  fancy quotes opening/.style={
    fancy quotes marks,
  },
  fancy quotes closing/.style={
    fancy quotes marks,
  },
  fancy quotes background/.style={
    show background rectangle,
    inner frame xsep=0pt,
    background rectangle/.style={
      fill=gray!25,
      rounded corners,
    },
  }
}

\newenvironment{fancyquotes}[1][]{%
\noindent
\tikzpicture[fancy quotes background]
\node[fancy quotes opening,anchor=north west] (fq@ul) at (0,0) {``};
\tikz@scan@one@point\pgfutil@firstofone(fq@ul.east)
\pgfmathsetmacro{\fq@width}{\linewidth - 2*\pgf@x}
\node[fancy quotes,#1] (fq@txt) at (fq@ul.north west) \bgroup}
{\egroup;
\node[overlay,fancy quotes closing,anchor=east] at (fq@txt.south east) {''};
\endtikzpicture}

\makeatother