\section{Limitations}
\author{Jesper}
\begin{frame}{Limitations}
	\framesubtitle{Deadlines}
	\begin{itemize}
		\item{Scalability}
		\item{Persistence modularity}
		\item{Large responses}
		\item{Error handling}
		\item{Polling}
		\item{Concurrency issues}
	\end{itemize}
\end{frame}

\subsection{Scale}

\begin{frame}{Limitations}
	\framesubtitle{Philosophy}
	\textbf{Enable scalability}, without inhibiting velocity,
	because scalability is \textbf{useless until} you need to scale.
\end{frame}

\begin{frame}{Limitations}
	\framesubtitle{Scaling the application}
	\begin{itemize}
		\item\textbf{Horizontally} separate the database
			\begin{itemize}
				\item Scale the database \textbf{Vertically}
			\end{itemize}
		\item \textbf{Horizontally} scale the frontend instances
			\begin{itemize}
				\item Additional configuration
				\item More \textbf{complexity} yields more problems
			\end{itemize}
		\item \textbf{Horizontally} scale the database
	\end{itemize}
\end{frame}

\begin{frame}{Limitations}
	\framesubtitle{Horizontal scalability}
	We suspect that we can scale to a \textbf{maximum realistic} number of users of our system. Although we haven't done the tested the claim.
\end{frame}

\begin{frame}{Limitations}
	\framesubtitle{Performance}
	Scalability is \textbf{nice}, but performance is \textbf{nicer}.
\end{frame}

\begin{frame}{Limitations}
	\framesubtitle{Database modularity}
	\begin{itemize}
		\item Hibernate ORM \textbf{could} be switched out
		\item Hibernate Search is \textbf{more difficult}
			\begin{itemize}
				\item Hibernate spatial might be a good alternative
			\end{itemize}
	\end{itemize}
\end{frame}

\subsection{Large responses}
\begin{frame}{Limitations}
	\framesubtitle{Large responses}
	\begin{itemize}
		\item Large datasets give rise to \textbf{large responses}
		\item Solvable by \textbf{pagination}
		\item Missing \textbf{elegant abstraction}
	\end{itemize}
\end{frame}

\subsection{Error handling}
\begin{frame}{Limitations}
	\framesubtitle{Error handling}
	\begin{itemize}
		\item HTTP error codes should mean that the response doesn't matter
		\item Out tools didn't agree
		\item Map exceptions to JSON on error
	\end{itemize}
\end{frame}

\begin{frame}{Limitations}
	\framesubtitle{Error handling}
	Errors should be formatted, they are responses too!
\end{frame}

\subsection{Polling}
\begin{frame}{Limitations}
	\framesubtitle{Long polling}
	\begin{itemize}
		\item \textbf{Active} polling
			\begin{itemize}
				\item Uses a lot of bandwidth
				\item Increasing poll interval leads to poor \textbf{immediacy}
			\end{itemize}
		\item \textbf{Long} polling
			\begin{itemize}
				\item Better \textbf{immediacy}
				\item Better \textbf{performance}
				\item Less \textbf{bandwidth} utilization
			\end{itemize}
	\end{itemize}
\end{frame}

\begin{frame}{Limitations}
	\framesubtitle{Long polling}
	Some data \textbf{naturally} propagates from the server to the client, rather than the inverse.
\end{frame}

\tikzset{
    invisible/.style={opacity=0.1},
    visible on/.style={alt=#1{}{invisible}},
    alt/.code args={<#1>#2#3}{%
      \alt<#1>{\pgfkeysalso{#2}}{\pgfkeysalso{#3}} % \pgfkeysalso doesn't change the path
    },
  }

\subsection{Concurrency}
\begin{frame}{Limitations}
	\framesubtitle{Concurrency}
	\begin{center}
	\begin{tikzpicture}[->,>=stealth',auto,node distance=3cm,
		thick,main node/.style={circle,draw}]

		\node[main node] (1) {Request 1};
		\node[main node] (2) [right of=1] {Server};
		\node[main node] (3) [right of=2] {Request 2};

		\path[every node/.style={font=\sffamily\small}]
			(1) edge[bend right, visible on=<1->] node [right] {} (2)
			(3) edge[bend left, visible on=<1->] node [right] {} (2)
			(2) edge[bend right, visible on=<2->] node [right] {} (1)
			(2) edge[bend left, red, visible on=<3->] node [right] {} (3);
	\end{tikzpicture}
	\end{center}
\end{frame}

\begin{frame}{Limitations}
	\framesubtitle{Concurrency}
	Databases do not automatically solve concurrency problems, and webservers are multi-threaded.
\end{frame}
