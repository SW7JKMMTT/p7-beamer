\section{Introduction}
\author{Thomas}
\subsection{Our idea}

\begin{frame}{Our idea}
    \framesubtitle{What led to the system}
    \begin{itemize}
        \item<1-> Our topic
        \item<2-> Problem statement

        \vspace{-12pt}
        \medskip
        {\addtolength{\leftskip}{2mm}\addtolength{\rightskip}{2mm}\noindent\hrulefill\it

        \noindent How can a \textbf{scalable} system be designed and implemented,
        such that \textbf{geospatial} and \textbf{relevant data} can be collected and aggregated in a \textbf{timely} way from a disjoint fleet of vehicles,
        while maintaining \textbf{arbitrary possibilities} of presenting the data to a given user?

        \vspace{-8pt}
        \noindent\hrulefill

        }
        \item<3-> Use case
        \begin{itemize}
            \item<3-> Lorry drivers
        \end{itemize}
        \item<4-> State of the art
        \begin{itemize}
            \item<4-> Fleetio
        \end{itemize}
    \end{itemize}
    \note<1-1>[item]{tracking of cars, get overview of fleet of vehicles, OBD, impractical to work with real car}
    \note<2-2>[item]{emphasise on requirements}
    \note<3-3>[item]{Different use cases, to narrow the area of interest: military, taxi, buses, lorry drivers}
    \note<3-3>[item]{We choose lorry drivers: easy info, comprehensible problem, had some interesting aspects: long hauls etc.}
    \note<4-4>[item]{sota, to see what was out there, lots of closed solutions}
    \note<4-4>[item]{found fleetio interesting, for small companies, lacks automation, full OBD advantage}

\end{frame}

\subsection{System overview}
\begin{frame}{System overview}
    \framesubtitle{Overview of the system}
    \begin{figure}
        \pgfdeclarelayer{background}
        \pgfsetlayers{background,main}
        \tikzstyle{bgbox}=[fill=green!20, rounded corners, draw=GoogleGreen, dashed]
        \center
        \begin{tikzpicture}[auto, node distance=1.5cm,
                stage/.style={rectangle, rounded corners, minimum width=1.5cm, minimum height=1cm,text centered, fill=diagramLight, draw, align=center},
                sharp/.style={rectangle, minimum width=9.0cm, minimum height=1cm,text centered, fill=diagramLight, draw, align=center},
                server/.style={rectangle, rounded corners, minimum width=1.5cm, minimum height=1cm,text centered, fill=diagramLight, rectangle split, rectangle split parts=2, draw},
                flow/.style={draw, ->},
                reverse_flow/.style={draw, <-},
                double_flow/.style={draw, <->},
                create/.style={draw, ->},
                use/.style={draw, dotted, ->}
            ]

            \node (backend) at (0,0) [draw=none] {};

            \node (web2) [draw=none, left=0.3cm of backend] {};

            \node (dataprov) [stage, left=0.3cm of web2] {Producer \\ client};
            \node (ghost1) [draw=none, left=0.7cm of dataprov] {};
            \node (car) [stage, above=0.4cm of ghost1] {Car};
            \node (sat) [stage, below=0.4cm of ghost1] {GPS \\ satellite};

            \node (web1) [draw=none, right=0.3cm of backend] {};

            \node (front) [draw=none, right=0.3cm of web1] {};

            \draw [double_flow] (dataprov) -| (sat);
            \draw [double_flow] (dataprov) -| (car);
        \end{tikzpicture}
    \end{figure}
    \note<1->[item]{\textbf{Producer client}, collection of data, different data sources, car, gps, camera etc.}
    \note<1->[item]{\textbf{Car}, OBD, other sources of data}
    \note<1->[item]{\textbf{GPS}, position data, time, speed}
\end{frame}

\begin{frame}{System overview}
    \framesubtitle{Overview of the system}
    \begin{figure}
        \pgfdeclarelayer{background}
        \pgfsetlayers{background,main}
        \tikzstyle{bgbox}=[fill=green!20, rounded corners, draw=GoogleGreen, dashed]
        \center
        \begin{tikzpicture}[auto, node distance=1.5cm,
                stage/.style={rectangle, rounded corners, minimum width=1.5cm, minimum height=1cm,text centered, fill=diagramLight, draw, align=center},
                sharp/.style={rectangle, minimum width=9.0cm, minimum height=1cm,text centered, fill=diagramLight, draw, align=center},
                server/.style={rectangle, rounded corners, minimum width=1.5cm, minimum height=1cm,text centered, fill=diagramLight, rectangle split, rectangle split parts=2, draw},
                flow/.style={draw, ->},
                reverse_flow/.style={draw, <-},
                double_flow/.style={draw, <->},
                create/.style={draw, ->},
                use/.style={draw, dotted, ->}
            ]

            \node (backend) at (0,0) [draw=none] {};

            \node (web2) [draw=none, left=0.3cm of backend] {};

            \node (dataprov) [draw=none, left=0.3cm of web2] {};
            \node (ghost1) [draw=none, left=0.7cm of dataprov] {};
            \node (car) [draw=none, above=0.4cm of ghost1] {};
            \node (sat) [draw=none, below=0.4cm of ghost1] {};

            \node (web1) [draw=none, right=0.3cm of backend] {};

            \node (front) [stage, right=0.3cm of web1] {Consumer \\ client};
        \end{tikzpicture}
    \end{figure}
    \note<1->[item]{\textbf{Consumer client}, different types, BI system, DW system, GUI of back--end, what needs the data}
\end{frame}

\begin{frame}{System overview}
    \framesubtitle{Overview of the system}
    \begin{figure}
        \pgfdeclarelayer{background}
        \pgfsetlayers{background,main}
        \tikzstyle{bgbox}=[fill=green!20, rounded corners, draw=GoogleGreen, dashed]
        \center
        \begin{tikzpicture}[auto, node distance=1.5cm,
                stage/.style={rectangle, rounded corners, minimum width=1.5cm, minimum height=1cm,text centered, fill=diagramLight, draw, align=center},
                sharp/.style={rectangle, minimum width=9.0cm, minimum height=1cm,text centered, fill=diagramLight, draw, align=center},
                server/.style={rectangle, rounded corners, minimum width=1.5cm, minimum height=1cm,text centered, fill=diagramLight, rectangle split, rectangle split parts=2, draw},
                flow/.style={draw, ->},
                reverse_flow/.style={draw, <-},
                double_flow/.style={draw, <->},
                create/.style={draw, ->},
                use/.style={draw, dotted, ->}
            ]
            %, line width=0.5mm, draw=black
            \node (backend) at (0,0) [stage] {Back--end};

            \node (web2) [draw, cloud, cloud puffs=12, cloud ignores aspect, minimum width=2.5em, minimum height=1.5em,fill=blue!20, left=0.3cm of backend] {};

            \node (dataprov) [stage, left=0.3cm of web2] {Producer \\ client};
            \node (ghost1) [draw=none, left=0.7cm of dataprov] {};
            \node (car) [stage, above=0.4cm of ghost1] {Car};
            \node (sat) [stage, below=0.4cm of ghost1] {GPS \\ satellite};

            \node (web1) [draw, cloud, cloud puffs=12, cloud ignores aspect, minimum width=2.5em, minimum height=1.5em,fill=blue!20, right=0.3cm of backend] {};

            \node (front) [stage, right=0.3cm of web1] {Consumer \\ client};

            \draw [double_flow] (backend) -- (web2);
            \draw [double_flow] (web2) -- (dataprov);

            \draw [double_flow] (dataprov) -| (sat);
            \draw [double_flow] (dataprov) -| (car);

            \draw [double_flow] (backend) -- (web1);
            \draw [double_flow] (web1) -- (front);
        \end{tikzpicture}
    \end{figure}
    \note<1->[item]{\textbf{Back--end}, combine producer and consumer, our focus, REST API, db}
    \note<1->[item]{normalise data at producer client, back--end don't about data source}
    \note<1->[item]{neutral data, different devices: tablets, web, etc.}
    \note<1->[item]{Troels talk more about the back--end.}
\end{frame}